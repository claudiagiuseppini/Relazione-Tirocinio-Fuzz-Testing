\chapter{Conclusione}

In questo lavoro di tesi sono stati introdotti i concetti fondamentali di errori software, sanitizers e fuzzing, necessari per comprendere le attività svolte durante il tirocinio. L’importanza di queste tecniche è confermata anche dal fatto che, nel solo 2020, software mal progettato ha generato perdite economiche per 2,08 miliardi di dollari negli Stati Uniti, a dimostrazione della rilevanza di strumenti per l’individuazione precoce dei bug \cite{ref32}.

L’attività sperimentale ha portato all’individuazione di un numero elevato di crash. Molti di essi si sono rivelati duplicati o non riproducibili, e sono quindi stati eliminati poiché non rappresentativi di vulnerabilità effettive. Successivamente, l’analisi ha permesso di filtrare anche i bug già noti agli sviluppatori, non rilevanti ai fini di una nuova segnalazione.

I bug rimanenti costituiscono il vero contributo di questo tirocinio: vulnerabilità concrete e potenzialmente sfruttabili da attaccanti. Ciò evidenzia come, all’interno di librerie e applicazioni largamente utilizzate a livello globale, possano ancora celarsi punti deboli in grado di avere conseguenze significative in termini di sicurezza.
Oltre a rilevare bug critici, abbiamo anche testato il sanitizer QMSan, provando che questo sia capace ed efficace nella rilevazione di UUM bugs. 
Il progetto ha inoltre evidenziato in maniera chiara l’importanza dell’impiego dei sanitizer all’interno delle campagne di test. Questi strumenti hanno consentito di individuare errori di natura semplice e apparentemente marginale, ma che sarebbero difficilmente rilevabili tramite una mera ispezione manuale del codice. La possibilità di intercettare tali anomalie si è rivelata cruciale non solo per incrementare l’affidabilità complessiva del software, ma anche per eliminare comportamenti indesiderati, con un conseguente miglioramento dell’esperienza dell’utente finale.

Va sottolineato che i risultati presentati sono frutto di un periodo limitato di tre mesi di lavoro. È quindi plausibile che ulteriori vulnerabilità restino da individuare. Per questo motivo, appare auspicabile un continuo sviluppo e consolidamento di piattaforme di sicurezza automatizzata come OSS-Fuzz, le quali svolgono un ruolo sempre più cruciale nella protezione degli utenti finali e nell’evoluzione del software open source.
